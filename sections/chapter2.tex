\section{Abstract Groups}

By denoting the elements of a group by certain symbols which obey a given rule
of multiplication, we obtain the so-called abstract group. In this chapter, we
will review some of its properties.

\subsection{Translation group} % (fold)
\label{sub:Translation group}

Suppose that the group $G$ consists of elements $g_1,g_2,\dots,g_m$. Let us
multiply its elements on the right by some element $g_\ell$, i.e., let's carry
a right translation along the group.

\begin{align}
    g_1g_\ell,g_2g_\ell,\dots,g_mg_\ell
\end{align}

We see that each group element is encountered once and only once in this sequence.
Suppose that an element could be encountered more than once such that $g_ig_\ell=
g_jg_\ell$, then this would imply that $g_i=g_j$ by the existence of the inverse.
Since there are as many elements as elements in the group, and they are all
unique, then because a group is closed this implies that all elements are
represented.
Let $g_\ell$ be an arbitrary element of the group, it is clear that
$\qty(g_\ell g^{-1}_\ell)g_\ell=g_\ell$, and, consequently, the element $g_\ell$
also appears in the sequence. Since the number of elements in our sequence is the
order of the group, each element can be found in the sequence only once. The
sequence of elements $g_\ell g_1,g_\ell g_2,\dots,g_\ell g_m$ has the same
properties.

% subsection Translation group (end)

\subsection{Subgroups} % (fold)
\label{sub:Sub-groups}

A set of elements belonging to a group $G$, to which itself forms a group with
the same multiplication rule, is a subgroup of $G$. The remainder of the group
$G$ cannot form a group since it will not contain the unit element.

% subsection Sub-groups (end)

\subsection{The order of an element} % (fold)
\label{sub:The order of an element}

Let us take an arbitrary element $g_i$ of a group $G$ and consider the powers of
$g_i,g_i^2,g_i^3,\dots$. Since we are considering a finite group, the members of
this sequence must appear repeatedly.

Suppose for example that
\begin{align}
    g_i^{k_1}=g_i^{k_2}=g_i\qquad k_2>k_1\\
    \rightarrow\ g_i^{k_2}=g_i^{k_1}g_i^{k2-k1}=g_ig_i^{k_2-k_1}=g_i\implies
    g_i^{k_2-k_1}=E.
\end{align}

The smallest exponent $h$ for which $g_i^h=E$ is the order of the element $g_i$.
The set of elements $g_i,g_i^2,\ddots g_i^h=E$ is the period of the element $g_i$.
It is clear that the period of an element of the group is a subgroup of $G$. All
the elements of this subgroup commute and consequently, the subgroup is Abelian.

If $h$ is the order of the element $g_j$, then $g_j^{h-1}=g_j^{-1}$,
therefore
for finite groups, the existence of the inverse is a consequence of the group
order properties.
\begin{proof}
    Recall the definition of a group.
    \begin{itemize}
        \item Composition law. $\forall a,b\in G, a\cdot b\in G$.
        \item Associativity. $\forall a,b,c\in G,
            (a\cdot b)\cdot c=a\cdot(b\cdot c)$.
        \item Neutral element. $\exists E\in G\ni \forall a\in G, E\cdot a=a\cdot E=a$.
        \item Symmetry. $\forall a\in G\exists a^{-1}\in G\ni
            a\cdot a^{-1}=a^{-1}\cdot a=E$.
    \end{itemize}
    The goal of this proof is to show that for finite groups, we can relax the
    symmetry axiom as it is a consequence of the group order properties. This
    section already proved most of this proof.
\end{proof}


% subsection The order of an element (end)

\subsection{Cosets} % (fold)
\label{sub:Cosets}

Let $H$ be a subgroup of a group $G$, $h_1,h_2,\ddots, h_m$, where $m$ is the
order of $H$. Let us now construct the following sequences of sets of elements
of $G$.
\begin{itemize}
    \item Let us take $g_1$ contained in $G$, which is not contained in $H$, and
        construct the set $g_1h_1,g_1h_2,\dots,g_1h_m$, which we denote by $g_1H$.
    \item Now we take another element of $G$, $g_2\notin H$, and set up $g_2H$.
    \item \vdots
\end{itemize}

As a result, we obtain the sequence $H,g_1H,g_2H,\dots,g_{k-1}H$. The sets $g_iH$
are the \emph{left cosets} of the subgroup $H$. We shall show that the cosets
defined above have no common elements. Let us suppose that the sets $g_1H$ and
$g_2H$ have one common element.

\begin{align}
    g_1h_1=g_2h_2\rightarrow g_2=g_1h_1h_2^{-2}=g_1h_3
\end{align}

so that $g_2$ belongs to the $g_1H$ set. But this conflicts with the original
assumption that each element of a group $G$ enters only one of the cosets. Since
$G$ contains $n$ elements, and each of the cosets contains $m$ elements, it
follows that $m=\frac nk$. The number $k$ is the index of the subgroup $H$. We
see that the order of the subgroup is a divisor of the order of the group.

Similarly, we can do the same with the right cosets, in constructing the cosets,
we have a choice in selecting the element $g_i$. It can be shown that for an
acceptable choice of the elements $g_i$, we can obtain the same set of cosets
and the same decomposition. This follows directly from the following theorem.

\begin{theorem}
    Two cosets $g_iH$ and $g_kH$ (being $g_i, g_k$ any two elements of $G$)
    either coincide or have no common elements. If these sets have at least
    one common element $g_ih_\alpha=g_kh_\beta\rightarrow g_k=g_\ell h_\alpha
    h_\beta^{-1}\implies g_k\in g_iH$.
    However, any element of the set $g_kH$ can be represented in the form
    $g_kh_j=g_ih_\alpha h_\beta h_\beta^{-1} h_j=g_ih_\gamma$ and will also belong
    to $g_iH$.
    The group $G$ can therefore uniquely decompose into (left or right) cosets
    of subgroup $H$.
\end{theorem}

% subsection Cosets (end)

\subsection{Conjugate elements and Class} % (fold)
\label{sub:Conjugate elements and Class}

Let $g$ be an element of the group $G$ and let us construct the element
$g'=g_igg_i^{-1},\ g_i\in G$. The lements $g$ and $g'$ are said to be conjugate.
Let us suppose that $g_i$ runs over all elements of the group $G$. We then obtain
$n$ elements, some of which might be equal. Let the number of distinct elements
be $k$, let us denote them by $g_1,g_2,\dots,g_k$. It is clear that this set
includes all the elements of the group $G$, which are conjugate to the element
$g$. All the elements of this set are mutually conjugate.

Let
\begin{align}
    g_1&=g_\alpha gg_\alpha^{-1}\\
    g_2&=g_\beta gg_\beta^{-1}.
\end{align}
Then we have
\begin{align}
    g&=g_{\alpha}^{-1}g_1g_\alpha\\
    g_2&=g_\beta g_\alpha^{-1}g_1g_\alpha g_\beta^{-1}=
    g_\beta g_\alpha^{-1}g_1\qty(g_\beta g_\alpha^{-1})^{-1}.
\end{align}
The set of all the mutually conjugate elements form \emph{a class}. Thus the
elements $g_1,g_2,\dots,g_k$ form a class of conjugate elements. The number of
elements in a class is its order. Any finite group can be denoted into a number
of classes of conjugate elements. The unit element of a group is a class by itself.
All the elements of a given class have the same order. The product of the elements
of two classes consists of whole classes.
\begin{proof}[]
    \begin{align}
        C_iC_j&=\sum_k hijk C_k
    \end{align}
    where $k$ is the multiplicity of the class.
    If $g_p\in C_iC_j$, then the entire class $C_p$ to which $g_p$ belongs
    itself belongs to the set $C_iC_j$. Let
    \begin{align}
        g_p=g_ig_j.
        \begin{cases}
            g_i\in C_i\\
            g_j\in C_j
        \end{cases}
    \end{align}
    We then have that for any $g\in G$,
    \begin{align}
        g^{-1}g_pg=g^{-1}g_igg^{-1}g_jg\in C_iC_j
    \end{align}
    it remains to show that each element of the class $C_p$ enters the set
    $C_iC_j$ the same number of times. Suppose that the element $g_p$ enters
    twice, $g_ig_j=g_p$ and $g'_ig'_j=g_p$, where $g_i\neq g'_i$ and
    $g_j\neq g'_j$. Each element $g'^{-1}g_pg'\ (g'\in G)$ will $C_p$ then be
    contained in $C_iC_j$ at least twice.
    \begin{align}
        g'^{-1}g_pg'&=g'^{-1}\qty(g_ig_j)g'=
        \qty(g'^{-1}g_ig')\qty(g'^{-1}g_jg')\\
        g'^{-1}g_pg'&=g'^{-1}\qty(g'_ig'_j)g'=
        \qty(g'^{-1}g'_ig')\qty(g'^{-1}g'_jg')
    \end{align}
    if we consider the result, it follows that $g'^{-1}g_ig'\neq g'^{-1}g'_ig'$.
    The element $g'^{-1}g_pg'$ will not be encountered more than twice, since
    otherwise $g_p$ would be encountered more than twice, which contradicts the
    original argument.
\end{proof}

% subsection Conjugate elements and Class (end)

\subsection{Invariant subgroup}
Let $H$ be a subgroup of the group $G$, and suppose that $g_i\in G$. Consider
the set of elements $g_iHg^{-1}_i$, where $g_i$ is fixed. That set is also
a group, since all the group axioms are satisfied for it.
If $g_i\in H$, then the similar subgroup will coincide with $H$. If $g_i\notin H$,
then in general we obtain a subgroup of which is different from $H$. When the
subgroup $H$ coincide with all its similar subgroups, it is called an invariant
subgroup or a normal divisor. An invariant subgroup is represented by the letter
$N$. It follows from this definition that if an invariant subgroup contains an
element $g\in G$, then it will also contain the entire class to which $g$
belongs. The invariant subgroup consists of the whole class of the group.\\

For an invariant subgroup of $G$, the left and right cosets coincide.
\begin{align}
    g_iN=g_iNg_i^{-1}g_i=Ng_i\qquad\text{since}\ g_iNg_i^{-1}=N.
\end{align}
Any group has two trivial invariant subgroups: the group itself and the unit
element. Groups which do not have invariant subgroups other that the trivial
ones are called simple.

It would be useful to prove that $NN=N$.

\subsection{The factor group} % (fold)
\label{sub:The factor group}

Let $N$ be an invariant subgroup of the group $G$. Let us decompose $G$ into the
cosets $N$,
\begin{align}
    N, g_1N, g_2N,\dots,g_{k-1}N
\end{align}
and lets form a set $g_1Ng_2N$, which consists of different elements
$g_1h_\alpha g_2h_\beta$, where $h_\alpha$ and $h_\beta$ run independently over
the entire subgroup $N$. We can see that
\begin{align}
    g_1Ng_2N=g_1g_2g_2^{-1}Ng_2N=g_1g_2NN=g_1g_2N=g_3N
\end{align}
if the set $g_1Ng_2N$ is called the product of sets $g_1N$ and $g_2N$, then the
product of two cosets of $N$ will again give a coset $N$, since the multiplication
of a coset of $N$ by $N$ on the right and left does not change these cosets.
\begin{align}
    Ng_1N=g_1g_1^{-1}Ng_1N=g_1NN=g_1N
\end{align}
For each coset $g_iN$, there is a coset $g_i^{-1}N$ such that the product is
equal to $N$.
\begin{align}
    g_i^{-1}Ng_iN=NN=N
\end{align}

The cosets of an invariant subgroup can be regarded as the elements of a new
group in which $N$ plays the role of the unit element. This group is what we call
\emph{the factor group} or \emph{the quotient group} of the invariant subgroup.
Its order is equal to the order of $N$.

% subsection The factor group (end)

\subsection{Isomorphism and Homomorphism of groups} % (fold)
\label{sub:Isomorphism and Homomorphism of groups}

If between the elements of two groups there is a one-to-one correspondence which
preserves group multiplication, then the group are \emph{isomorphic}. Let $G$ and
$\widetilde{G}$ be two isomorphic groups, then if elements $g_i$ and $g_k$ of $G$
correspond to the elements $\widetilde{g}_i$ and $\widetilde{g}_k$ of
$\widetilde{G}$, i.e., $g_i\leftarrow\widetilde{g}_i$,
$g_k\leftarrow\widetilde{g}_k$, then
$g_1=g_ig_k\rightarrow\widetilde{g}_1=\widetilde{g}_i\widetilde{g}_k$. We can
reduce the investigation of a given group to that of another group isomorphic to
it.

If to each element of $G$ there corresponds only one definite element of
$\widetilde{G}$, and to each element of $\widetilde{G}$ a number of elements of
$G$, and this correspondence is preserved under group multiplication,
$\widetilde{G}$ if \emph{homomorphic} to $G$. Homomorphism has the following
properties:
\begin{itemize}
    \item If $\widetilde{G}$ is homomorphic to $G$, then the element of G
        corresponds to the unit element of $\widetilde{G}$.
    \item The mutually reciprocal element of $G$ correspond to the mutually
        reciprocal elements of $\widetilde{G}$.
        \begin{align}
            g_1g_k=E\qquad \widetilde{g}_i\widetilde{g}_k=\widetilde{E}.
        \end{align}
    \item All elements of $G$, to which corresponds to the unit element of
        $\widetilde{G}$ form an invariant subgroup $N$ of the group $G$. Elements
        $g'_1,g'_2,\dots,g'_j$ of $G$ correspond to $\widetilde{E}$ of
        $\widetilde{G}$. The product $g'_ig'_k$ correspond to
        $\widetilde{E}\widetilde{E}=\widetilde{E}$. $g'_ig'_k=g'_\ell$ and the
        set $g'_1,g'_2,\dots,g'_j$ is closed with respect to group multiplication.
\end{itemize}
from (a), it contains a unit element, but since $\widetilde{E}$ is the inverse
of itself and because of (b) for each element $g'_\ell$ we can find the inverse.
From $\widetilde{g}\widetilde{E}\widetilde{g}^{-1}=\widetilde{E}$, where
$\widetilde{g}$ is an arbitrary element of $\widetilde{G}$, it follows that
$gg_ig^{-1}=g'_f$ for an arbitrary element of $G$.
\begin{align}
    g'_1,g'_2,\dots,g'_s
\end{align}
forms an \emph{invariant} subgroup of $G$.

% subsection Isomorphism and Homomorphism of groups (end)

\subsection{Exercices} % (fold)
\label{sub:Exercices}

Consider the group $S_6$, symmetric group of order $6$, with the elements
$\{E, A, B, C, D, F\}$.
\begin{center}
\begin{tabular}{ c|c c c c c c }
  & E & A & B & C & D & F \\
 \hline
    E & E & A & B & C & D & F \\
    A & A & E & D & F & B & C \\
    B & B & F & E & D & C & A \\
    C & C & D & F & E & A & B \\
    D & D & C & A & B & F & E \\
    F & F & B & C & E & E & D
\end{tabular}
\end{center}

\begin{enumerate}
    \item Find the order of all the elements.
    \item Find the possible subgroups.
    \item Divide the group into cosets, and verify that this can be done in an
        unique way.
    \item Divide the group into classes of conjugate elements.
    \item Find the invariant subgroups and verify that the left and right
        cosets are the same.
    \item Write down the multiplication table for the corresponding factor
        groups.
    \item Show that the abstract groups has the following realizations

        \begin{enumerate}
            \item Permutation group of $3$ elements.
            \item Matrix group of order $2$: $S_6$ can also be 2x2 matrices
                corresponding to rotations and reflections.
        \end{enumerate}
\end{enumerate}

% Q1
Recall the definition of the order of an element
(\autoref{sub:The order of an element}).
Then we can compute for all elements the smallest exponent $h$ such that $g^h=E$.
From the table of multiplication, we can see that the order of all the elements
except for the unit element and $F$ and $D$ is $2$. The order of the unit
element is $1$. The order of $D$ and $F$ is $3$.

% Q2
Recall the definition of the subgroup (\autoref{sub:Sub-groups}). Then the
subgroups are
\begin{align*}
    &\{E\}\\
    &\{A,E\},\{B,E\},\{C,E\}\\
    &\{D,F,E\}\\
    &\{A,B,C,D,E,F\}
\end{align*}

% Q3
Recall the definition of the cosets (\autoref{sub:Cosets}). Let's start with
$H=\{E\}$. The left and right cosets are then equal and form $G$. Let's continue
with $H=\{A,E\}$, we then have the left cosets $\{D,B\}, \{F,C\}$.

% subsection Exercices (end)
