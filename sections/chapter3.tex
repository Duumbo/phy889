\section{Representations of point groups} % (fold)
\label{sec:Representations of point groups}

Consider a finite group $G$ with elements $g_1,g_2,\dots,g_m$. If a group $\hat{T}$
of linear generators $\hat{T}g_i$ in space $R$ is homomorphic to $G$, then the
group $\hat{T}$ is said to form a representation of $G$.
Homomorphism leads to $\hat{T}g_i\cdot\hat{T}g_k=\hat{T}g_ig_k$. If $R$ is the
$n$-dimensional vector space $R_n$, then any of its elements $x$ can be expanded
in terms of $n$ unit vector $\vb{e}_k$ forming the basis of this space.
\begin{align}
    x&=x_1\vb{e}_1+x_2\vb{e}_2+\dots+x_n\vb{e}_n
\end{align}

The operator $\hat{T}g_i$ will be defined if we specify its effect on each
of the unit vectors $\vb{e}_k$. Suppose that

\begin{align}
    \hat{T}g_i\vb{e}_k&=\sum_{r=1}^nD_{r,k}(g_i)\vb{e}_r.
\end{align}

To each element $g_i$ of our group, we can assign a matrix $\norm{D_{r,k}(g_i)}$.
It is also clear that the unit element of the group can be associated with a unit
matrix, and the inverse elements can be associated with inverse matrices.

Let us show now that for the matrices $D$ we have
\begin{align}
    D(g_i)D(g_j)=D(g_ig_j)
\end{align}
If we apply $\hat{T}g_i$ and $\hat{T}g_j$ successively to the unit vector
$\vb{e}_k$, we obtain
\begin{align}
    \hat{T}g_i\hat{T}g_j\vb{e}_k&=\hat{T}g_i\sum_rD_{r,k}(g_i)\vb{e}_r\\
                                &=\sum_{f,r}D_{r,k}(g_j)D_{f,r}(g_i)\vb{e}_f.
\end{align}
On the other hand
\begin{align}
    \hat{T}g_i\hat{T}g_i\vb{e}_k&=\hat{T}_{g_ig_j}\vb{e}_k\\
                                &=\sum_fD_{f,k}(g_ig_j)\vb{e}_f.
\end{align}
So $D(g_i)D(g_j)=D(g_ig_j)$ is valid. We will show now that the matrices $D(g_i)$
form a representation of order $n$ of the group $G$. The space $R_n$ is the
representation space, and the basis of this space is the basis of the
representation. By operating with $\hat{T}g_i$ on an arbitrary vector $\vb{x}$ of
space $R_n$, we obtain
\begin{align}
    \hat{T}g_i\vb{x}&=\sum_kx_k\hat{T}g_i\vb{e}_k\\
                    &=\sum_{k,r}x_kD_{r,k}(g_i)\vb{e}_r\\
                    &=\sum_rx'_r\vb{e}_r
\end{align}
where $x'_r=\sum_kD_{r,k}(g_i)x_k$. Let us consider now the change in the
representation matrix which occurs when a new basis $\vb{e}'_i$ is taken in the
space $R_n$, where the new basis is related to $\vb{e}_k$ by the linear
transformation
\begin{align}
    \vb{e}'_i&=\sum_k V_{ki}\vb{e}_k\label{star}\\
    \vb{e}_i&=\sum_k\{V^{-1}\}_{k,i}\vb{e}'_k.
\end{align}
Let us now apply $\hat{T}g_i$ to $\vb{e}'_i$ by using (\autoref{star})
\begin{align}
    \hat{T}g_i\vb{e}'_i&=\sum_kV_{ki}\hat{T}g_i\vb{e}_k\\
                       &=\sum_{k,s}V_{k,i}D_{s,k}(g_i)\vb{e}_s\\
                       &=\sum_{k,s,r}V_{k,i}D_{s,k}\{V^{-1}\}_{r,s}\vb{e}'_r\\
                       &=\sum_r \{V^{-1}DV\}_{r,i}\vb{e}'_r
\end{align}
Thus the representation matrices undergo a similar transformation when we
transform to the new basis. The representation by the matrices $V^{-1}DV$ is
equivalent to the representation by matrices $D$. If the representation
matrices are all unitary, the representation is said to be unitary.

\subsection{Examples of representation} % (fold)
\label{sub:Examples of representation}

\begin{itemize}
    \item The trivial representation, which is associated with the unit matrix.
    \item If the group elements are linear transformations, the matrices of these
        transformations themselves form a representation which is isomorphic to
        the group.
\end{itemize}
These are the trivial invariant subgroups.

\subsubsection{Quadratic Form}
Let's consider the derivation of one of the representations of the group
$C$ of matrices of linear transformations of invariables
$x_1,x_2,\dots,x_n$, $x'_i=\sum_kC_{i,k}x_k$. But now we are also going
to consider the quadratic form
\begin{align}
    \sum_{i,k}a_{i,k}x_ix_k\qquad a_{i,k}=a_{k,i}
\end{align}
Transformations of the variables $x_i$ induces a transformation of the
coefficients of this form. If we substitute
$x'_j=\sum_s\{C^{-1}\}_{j,s}x'_s$, we obtain the following quadratic form
\begin{align}
    \sum_{i,k,j,\ell}a_{i,k}\{C^{-1}\}_{i,j}x'_j\{C^{-1}\}_{k,\ell}x'_\ell
    &=\sum_{j,\ell}a'_{j,\ell}x'_jx'_\ell,
\end{align}
where
\begin{align}
    a'_{j,\ell}&=\sum_{i,k}a_{i,k}\{C^{-1}\}_{i,j}\{C^{-1}\}_{k,\ell}
\end{align}
If we use the notation $A=\norm{a_{i,k}}$, we can write down the transformation
rule for the coefficients $a_{i,k}$ in the matrix form $A'=C^{-1}AC$,
where $C^{-1}$ is the transpose of $C$. Let's now apply the
transformations $C_1$ and $C_2$ to $x_1,x_2,\dots,x_n$.
\begin{align}
    A''&=C_2^{-1}A'C_2=C_2^{-1}C_1^{-1}AC_1C_2\\
    A''&=\qty(C_2C_1)^{-1}A\qty(C_2C_1)
\end{align}
Applications of $C_1$ and $C_2$ is equivalent to $C_2C_1$. The
transformation of the coefficients of the \emph{quadratic form}
form themselves a representation.

% subsection Examples of representation (end)

\subsubsection{Schrödinger equation and its eigenfunctions} % (fold)
\label{sub:Schrödinger equation and its eigenfunctions}

Let's consider a quantum mechanical system described by the Schrödinger equation

\begin{align}
    \qty(\frac{-\hbar^2}{2m}\nabla^2+V(\vb{r}))\psi(\vb{r})=E\psi(\vb{r})
\end{align}
We shall assume that the symmetry group for this system consists of orthogonal
transformation $u_s$, defined by
\begin{align}
    \vb{r}'=u_s\vb{r}\\
    \vb{r}=u_s^{-1}\vb{r}'
\end{align}
Since the laplace operator is invariant under any orthogonal transformation of
coordinates, this yields to
\begin{align}
    \qty(-\frac{\hbar^2}{2m}\nabla^2+V\qty(u_s^{-1}\vb{r}))\psi(u_s^{-1}\vb{r})=E\psi
    \qty(u_s^{-1}\vb{r})
\end{align}
Moreover since the Schrödinger equation is invariant under the $u_s$ transformation
we must have
\begin{align}
    V\qty(u_s^{-1}\vb{r})V(\vb{r})
\end{align}
and therefore, the transformed wave-function
\begin{align}
    \psi'(\vb{r})&=\hat{T}_{u_s}\psi(\vb{r})=\psi(u_s^{-1}\vb{r})
\end{align}
is also an eigenfunction of the Schrödinger equation with the same eigenvalue.
Let's now consider $\psi_1(\vb{r}),\dots,\psi_k(\vb{r})$, a complete set of
orthonormal eigenfunctions of this equation, corresponding to eigenvalue $E$. We
will see that these functions form a basis of a group representation.
Each of the transformed functions can be written in the form
\begin{align}
    \hat{T}_{u_s}\psi_i(\vb{r})=\psi_i\qty(u^{-1}_s\vb{r})=\sum_{j=1}^kD_{j,k}
    (u_s)\psi_j(\vb{r})
\end{align}
$\hat{T}_{u_s}\psi_i(\vb{r})$ must also be orthonormal, since a change of the variables through an orthonormal transformation conserves the
orthonormalisation condition.
\begin{align}
    \int\psi_i(u_s\vb{r}')\psi_j(u_s\vb{r}')\dd{\vb{r}}=
    \int\psi_i(\vb{r})\psi_j(\vb{r})\dd{\vb{r}}
\end{align}
Matrices $\norm{D_{i,j}(u_s)}$ should be unitary, and hence, to each
transformation $u_s$ form the symmetry group of the Schrödinger equation, we
can assign a unitary matrix of order $k$. Let us consider $u_s$ and $u_t$ to be
transformations in the group. Their successive applications
\begin{align}
    \hat{T}_{u_s}\hat{T}_{u_t}\psi_i(\vb{r})&=
    \hat{T}_{u_s}\psi_i(u^{-1}_t\vb{r})\\
                                            &=\psi_i(u^{-1}_tu^{-1}_s\vb{r})\\
                                            &=\psi_i\qty((u_su_t)^{-1}\vb{r})\\
                                            \sum_{\ell=1}^kD_{\ell,\ell}(u_su_t)\psi_\ell(\vb{r}).
\end{align}
On the other hand
\begin{align}
    \hat{T}_{u_s}\hat{T}_{u_t}\psi_i(\vb{r})&=\hat{T}_{u_s}\sum_{j=1}^kD_{j,i}(u_t)\psi_j(\vb{r})\\
                                            &\sum_{j=1}^kD_{j,i}(u_t)\sum_{\ell=1}^kD_{\ell,j}(u_s)\psi_\ell(\vb{r})\\
                                            &=\sum_{\ell=1}^k\{D(u_s)D(u_t)\}_{\ell,i}\psi_\ell(\vb{r}).
\end{align}
$\implies\ $ With each energy eigenvalue, we can associate a representation and
establish the possible types of symmetry of the wave functions without
explicitly solving the Schrödinger equation.
% subsection Schrödinger equation and its eigenfunctions (end)

\subsection{Existence of an equivalent unitary representation} % (fold)
\label{sub:Existance of an equivalent unitary representation}

We shall show that any representation of a finite group is equivalent to a
unitary representation.
Suppose that we have a representation $D$ of the group $G$ consisting of the
elements $g_1,g_2,\dots,g_m$. We regard the representation matrices $D(g_i)$
as the transformation matrices in a $n$-dimensionnal space $R_n$
$\vb{x}(x_1,x_2,\dots,x_n)$ and $\vb{y}(y_1,y_2,\dots,y_n)$ are vectors in this space.
The scalar product will be defined
\begin{align}
    (\vb{x},\vb{y})=x_1y_1+x_2y_2+\dots+x_ny_n
\end{align}
and $D(g_i)$ transforms the vector $\vb{x}$ into the vector $\vb{x}^{(i)}$
\begin{align}
    \vb{x}^{(i)}=D(g_i)\vb{x}\\
    x_\alpha^{(i)}=\sum_\beta D_{\alpha,\beta}(g_i)x_\beta.
\end{align}
Let's suppose that $D(g_i)$ is not unitary, and it does not conserve the scalar
product. We shall show that it is possible to choose a new basis in $R_n$ such
that the transformation matrices for vector components will be unitary.
We take the average of the scalar product over the group and construct the
expression
\begin{align}
    \sum_{i=1}^m\qty(D(g_i)\vb{x},D(g_i)\vb{y})&=\sum_{i=1}^m\qty(\vb{x}^{(i)},\vb{y}^{(i)}).
\end{align}
This can be written as
\begin{align}
    \sum_{i=1}^m(\vb{x}^{(i)},\vb{y}^{(i)})=\qty(L\vb{x},L\vb{y})
\end{align}
where $L$ is a linear transformation.
\begin{align}
    \sum_{i=1}^m\qty(D(g_i)\vb{x},D(g_i)\vb{y})=\qty(\sum_{i=1}^mD^\dagger(g_i)D(g_i)\vb{x},\vb{y})
\end{align}
The matrix $D^\dagger(g_i)D(g_i)$ is Hermitian, and it can be reduced to a
diagonal form through a unitary transformation V
\begin{align}
    d&=V^{-1}\sum_{i=1}^mD^\dagger(g_i)D(g_i)V.
\end{align}
If we substitute $D(g_i)=V^{-1}D(g_i)V$, we can write
\begin{align}
    d&=\sum_{i=1}^mV^{-1}D^\dagger(g_i)VV^{-1}D(g_i)V\\
     &=\sum_{i=1}^m\widetilde{D}^\dagger(g_i)\widetilde{D}(g_i)
\end{align}
and the diagonal elements of the matrices $d$ are given by
\begin{align}
    d_{\alpha,\alpha}&=\sum_{i=1}^m\sum_{\beta=1}^n\widetilde{D}^{\dagger}_{\alpha,\beta}(g_i)\widetilde{D}_{\alpha,\beta}(g_i)\\
                     &=\sum_{i=1}^m\sum_{\beta=1}^n\abs{\widetilde{D}_{\alpha,\beta}(g_i)}^2.
\end{align}
Let us now determine the diagonal matrix $d^{1/2}$, whose  elements are
$\{d^{1/2}\}_{\alpha,\alpha}=\sqrt{d_{\alpha,\alpha}}$.
So $d^{1/2}d^{1/2}=d$ and if we use the self = conjoint property of $d^{1/2}$, we
have
\begin{align}
    \sum_{i=1}^m\qty(\vb{x}^(i),\vb{y}^(i))&=\qty(VdV^{-1}\vb{x},\vb{y})\\
                                           &=\qty(d^{1/2}d^{1/2}V^{-1}\vb{x},V^{-1}\vb{y})\\
                                           &=\qty(d^{1/2}V^{-1}\vb{x},d^{1/2}V^{-1}\vb{y}).
\end{align}
We can go to
\begin{align}
    \qty(L\vb{x},L\vb{y})\rightarrow L=d^{1/2}V^{-1}.
\end{align}
We can also show that the representation of $G$ given by the matrices $LDL^{-1}$
is unitary. For an arbitrary element $g_k$ of $G$,
\begin{align}
    \qty(LD(g_k)\vb{x},LD(g_k)\vb{y})=\qty(L\vb{x},L\vb{y}).
\end{align}
Prove it for next class.

% subsection Existance of an equivalent unitary representation (end)

\subsection{Reducible and irreducible representations} % (fold)
\label{sub:Reducible and irreducible representations}

Suppose that a representation $D$ of the group $G$ is given in a space $R_n$. If
in the space $R_n$ there is a subspace $R_k$, with $k<n$, which is invariant
under all transformations $D$, i.e., if for $x\in R_k$ we have a $D_x\in R_k$,
the representation is \emph{reducible}. Let us take the first $k$ unit vectors
in the space $R_n$ as the unit vectors of the subspace $R_k$. The representation
matrix must have the following form:
\begin{align}
    \mqty(D_{11} & D_{12} & \dots & D_{jk} & D_{1 k+1} & \dots & D_{1n}\\
    \vdots & \vdots & \ddots & \vdots & \vdots & \ddots & \vdots \\
    D_{k1} & D_{k2} & \dots & D_{kk} & D_{kk+1} & \dots & D_{kn}\\
    0 & \dots & \dots & 0 & D_{k+1 k+1} & \dots & D_{k+1n}\\
    \vdots & \vdots & \ddots & \vdots & \vdots & \ddots & \vdots\\
    0 & \dots & \dots & 0 & D_{nk+1} & \dots &D_{nn}
    )
\end{align}

If on the other hand we cannot define an invariant subspace in $R_n$,
the representation is \emph{irreducible}. We shall show that if a reducible
representation $D$ is unitary, the orthogonal complement of the subspace $R_k$,
which we denote $R_{a-k}$, is also invariant under the transformation of $D$.

Let us consider $x\in R_k,y\in R_{n-k}$ and $(x,y)=0$. Since the subspace $R_k$
is invariant $(D_g(x),y)=0)$, but as we know

\begin{align}
    \qty(D(g)x,y)&=\qty(x,D^\dagger(g)y)=(x,D^{-1}(g)y)\\
                 &=(x,D(g^{-1})y)=0\label{starrr},
\end{align}
and hence $D(g^{-1})y\in R_{n-k}$. When $g$ runs over the entire group,
the inverse element $g^{-1}$ will also do so. So (\autoref{starrr}) is satisfied
for all matrices of the representation in question, and the invariance $R_{n-k}$
is proved.

If we take the unit vectors of the subspace $R_k$, as the first $k$ vectors, and 
the remaining $n-k$ unit vectors as the vectors of the subspace $R_{n-k}$, the
representation matrix will be quasi-diagonal.

\begin{align}
    \mqty(D_{11} & D_{12} & \dots & D_{jk} & 0 & \dots & 0\\
    \vdots & \vdots & \ddots & \vdots & \vdots & \ddots & \vdots \\
    D_{k1} & D_{k2} & \dots & D_{kk} & 0 & \dots & 0\\
    0 & \dots & \dots & 0 & D_{k+1 k+1} & \dots & D_{k+1n}\\
    \vdots & \vdots & \ddots & \vdots & \vdots & \ddots & \vdots\\
    0 & \dots & \dots & 0 & D_{nk+1} & \dots &D_{nn}
    )
\end{align}

If the space $R$ can be resolved into invariant subspaces, in each of them an
irreducible representation is realised, then the representation $D$ is
\emph{fully reducible}. With a suitable choice of unit vectors, the matrix of
this representation is block diagonal. From this discussion, it follows that

\begin{itemize}
    \item A unitary representation of a group is always irreducible or fully
        reducible.
    \item Any representation of a finite group is either again irreducible or
        fully reducible.
\end{itemize}

If a representation $D$ is reducible, its matrices can be reduced to a
quasi-diagonal form by going under the new system of unit vectors.
We note that in this case, the representation matrices undergo the similarity
transformation: $D\rightarrow V^{-1}DV$, where $V$ is the unitary matrix relating
the unit vectors of the old and new basis.

A representation $D$ is reducible if there exists a non-singular matrix $V$
such that $V^{-1}DV$ is block-diagonal.

% subsection Reducible and irreducible representations (end)
% section Representations of point groups (end)
