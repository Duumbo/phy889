\section{Schur's Lemmas}
\subsection{Schur's First Lemma} % (fold)
\label{sub:Schur's First Lemma}

\begin{epigraph}
    A matrix which commutes with all the matrices of an irreducible\\
    representation is a multiple of the unit matrix.
\end{epigraph}

Let $D(y)$ represent the matrices of an irreducible representation of order $n$
of the group $G$, $g\in G$. Let's suppose the matrix $M$ commutes with all the
matrices  $D(g)MD(g)=D(g)M$. Let $R_n$ represent the space in which $D(g)$ is
realised. In this space, there should at least be one eigenvector of $M$, we
denote it $x$, such that $M\bar{x}=\lambda\bar{x}$. If we apply the transformation with the representation matrix $D(g)$ to the vector $x$ will have that
\begin{align}
    D(g)\bar{x}=\bar{x}g.
\end{align}
This resulting vector is also an eigenvector of $M$ with the same eigenvalue
$\lambda$.
\begin{align}
    M\bar{x}_g&=MD(g)x=\lambda D(g)\bar{x}\\
              &=\lambda\bar{x}_g.
\end{align}
It follows that the space of eigenvectors of the matrix $M$ corresponding to
the same eigenvalue is invariant under the transformation $D(g)$. But since $D(g)$
is irreducible, it follows that this subspace should coincide with the entire
$R_n$, and the matrix $M$ multiplying by any vector of the space $R_n$ by
$\lambda$, should be of the form
\begin{align}
    M=\mqty(\lambda&\dots&0\\0&\ddots&0\\0&\dots&\lambda)
\end{align}
If a representation is fully reducible, i.e., its matrices have the quasi-diagonal
form, one will always find a matrix which is not a multiple of the unit matrix
and which commutes with all the matrices of this representation. Once we verify
that this matrix can be taken to be the diagonal matrix in which diagonal element
corresponding to the different blocks are not equal to one another.
% subsection Schur's First Lemma (end)

\subsection{Schur's Second Lemma} % (fold)
\label{sub:Schur's Second Lemma}
\begin{theorem}
    Let $D^{(j)}(g)$ and $D^{(i)}(g)$ be the matrices of two irreducible non-equivalent
    representations of a group $G$ of order $n_1$ and $n_2$, respectively. Then, any
    rectangular matrix $M$ with $n_1$ columns and $n_2$ rows which satisfies this
    equation
    \begin{align}
        MD^{(i)}(g)=D^{(i)}(g)M,\qquad\forall g\in G
    \end{align}
    is a null matrix.
\end{theorem}

\begin{proof}
Let us take the Hermitian conjugate on both sides
\begin{align}
    D^{(i)\dagger}(g)M^\dagger=M^\dagger D^{(i)\dagger}(g)
\end{align}
$D^{(i)}(g)$ and $D^{(j)}(g)$ are unitary.
\begin{align}
    D^{(1)-1}(g)M^\dagger=M^\dagger D^{(2)-1}(g)\\
    D^{(1)}(g^{-1})M^\dagger=M^\dagger D^{(2)}(g^{-1})
\end{align}
If $g$ runs over the entire group, then $g^{-1}$ also do so. (\autoref{star})
can also be re-written as
\begin{align}
    D^{(1)}(g)M^\dagger=M^\dagger D^{(2)}(g).
\end{align}
Let's multiply both sides on the left by the matrix $M$.
\begin{align}
    MD^{(2)}M^\dagger&=MM^\dagger D^{(2)}(g)\\
    D^{(2)}(g)MM^\dagger&=MM^\dagger D^{(2)}(g)
\end{align}
According to Schur's first lemma, we conclude that $MM^\dagger$ must be a
multiple of the unit matrix.
\begin{align}
    MM^\dagger&=\lambda E_{n_2}
\end{align}
First case: $n_1=n_2$. In this case, $M$ must be singular $\det M=0$.
If this was not the case, the $MD^{(2)}(g)=D^{(2)}(g)M$ should yield the
condition for the equivalence of the representation
\begin{align}
    D^{(1)}=M^{-1}D^{(2)}M.
\end{align}
But this would mean $D^{(1)}$ and $D^{(2)}$ are isomorphic, which is not true
by hypothesis. $M$ cannot be invertible $\implies\det M=0$.
If we take
\begin{align}
    MM^\dagger=\lambda E_{n_2}\\
    \lambda=\sum_j M_{ij}\bar{M}_{ij}\\
    \det M\det M^\dagger=\lambda^{n_2}\\
    \lambda=0=\sum_j\abs{M_{ij}}^2\implies M_{ij}=0
\end{align}
In the other case, $n_2>n_1$. We augment the matrix $M$ so that it becomes a
square matrix with $n_2n_1$ rows and columns. We do the same to $M^\dagger$.
If we denote the two matrices $\widetilde{M}$ and $\widetilde{M}^\dagger$,
\begin{align}
    \widetilde{M}\widetilde{M}^\dagger = \lambda E_{n_2}\\
    \det\widetilde{M}^\dagger=\det\widetilde{M}=0\implies M_{ij}=0.
\end{align}
\end{proof}

% subsection Schur's Second Lemma (end)

\begin{exercise}
    Prove that any representation of a simple group (one without a normal divisor)
    is isomorphic to the group
    itself.
\end{exercise}

\begin{exercise}
    Use Schur's first lemma to show that all irreducible representations of an
    Abelian group are of order $1$.
\end{exercise}

\begin{exercise}
    Use Schur's first lemma to show that the sum of irreducible representation
    matrices corresponding to the elements of a given class is a multiple of the
    unit matrix.
\end{exercise}
